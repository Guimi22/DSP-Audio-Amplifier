
\chapter*{Resum}
\par Aquest treball de final de grau té com a objectiu el disseny i implementació d'un sistema de processament digital i amplificació d'àudio utilitzant una placa de desenvolupament d'una FPGA. El projecte es centra a desenvolupar un amplificador d'àudio de tipus D, conegut per la seva alta eficiència energètica, integrat amb un sistema de processament digital d'àudio (DSP).

\par Mitjançant l'ús d'una FPGA, s'exploren les possibilitats d'un processament digital altament personalitzable i de baix consum. L'estudi inclou el disseny de l'etapa de filtratge i modelat de soroll, la programació de la FPGA, i la validació del sistema complet, demostrant la viabilitat tècnica i pràctica de la proposta. Aquest treball posa en evidència el potencial de les FPGAs com a plataformes versàtils per al desenvolupament de sistemes d'àudio moderns.

\par En el present treball s'ha aconseguit implementar a la FPGA un bloc receptor I2S, una etapa de filtratge i sobremostreig de OSR = 32 i un modulador $\Sigma \Delta$ de segon ordre, donant com a resultat que el senyal d'àudio de l'entrada es pugui reproduir mitjançant l'ús d'uns auriculars. 
\addcontentsline{toc}{chapter}{\protect\numberline{}Resum}


\chapter*{Resumen}
\par Este trabajo de fin de grado tiene como objetivo el diseño e implementación de un sistema de procesamiento digital y amplificación de audio utilizando una placa de desarrollo con una FPGA. El proyecto se centra en desarrollar un amplificador de audio de tipo D, conocido por su alta eficiencia energética, integrado con un sistema de procesamiento digital de audio (DSP).

\par Mediante el uso de una FPGA, se exploran las posibilidades de un procesamiento digital altamente personalizable y de bajo consumo. El estudio incluye el diseño de la etapa de filtrado y modelado de ruido, la programación de la FPGA y la validación del sistema completo, demostrando la viabilidad técnica y práctica de la propuesta. Este trabajo pone de manifiesto el potencial de las FPGAs como plataformas versátiles para el desarrollo de sistemas de audio modernos.

\par En el presente trabajo se ha logrado implementar en la FPGA un bloque receptor I2S, una etapa de filtrado y sobremuestreo con OSR = 32, y un modulador $\Sigma \Delta$ de segundo orden, resultando en que la señal de audio de salida se reproduzca a través de unos auriculares.

\addcontentsline{toc}{chapter}{\protect\numberline{}Resumen}


\chapter*{Abstract}
\par This final degree project aims to design and implement an audio processing and amplification system using an FPGA. The project focuses on developing a Class D audio amplifier, known for its high energy efficiency, integrated with a digital signal processing (DSP) audio system.  

\par By leveraging the capabilities of an FPGA, the study explores the possibilities of highly customizable and low-power digital processing. The research includes the design of the filtering and noise-shaping stages, the programming of the FPGA, and the validation of the complete system, demonstrating the technical and practical feasibility of the proposal. This work highlights the potential of FPGAs as versatile platforms for the development of modern audio systems.

\par In the present work, an I2S receiver block, a filtering and oversampling stage with OSR = 32, and a second-order $\Sigma \Delta$ modulator have been successfully implemented on the FPGA resulting in the output audio signal being played through headphones.

\addcontentsline{toc}{chapter}{\protect\numberline{}Abstract}